% !TEX TS-program = xelatex
% !TEX encoding = UTF-8 Unicode

% This is a simple template for a LaTeX document using the "article" class.
% See "book", "report", "letter" for other types of document.

\documentclass[11pt]{book} % use larger type; default would be 10pt

\usepackage{ctex}
\usepackage{amsmath}
\usepackage{amsthm}
\usepackage{amssymb}

\usepackage{upgreek}
\usepackage{xltxtra}

\usepackage{geometry}

\usepackage{paralist}



\usepackage[utf8]{inputenc} % set input encoding (not needed with XeLaTeX)

%%% Examples of Article customizations
% These packages are optional, depending whether you want the features they provide.
% See the LaTeX Companion or other references for full information.
\usepackage{makeidx}


%%% PAGE DIMENSIONS
\usepackage{geometry} % to change the page dimensions
\geometry{a4paper} % or letterpaper (US) or a5paper or....
% \geometry{margin=2in} % for example, change the margins to 2 inches all round
% \geometry{landscape} % set up the page for landscape
%   read geometry.pdf for detailed page layout information

\usepackage{graphicx} % support the \includegraphics command and options

% \usepackage[parfill]{parskip} % Activate to begin paragraphs with an empty line rather than an indent

%%% PACKAGES
\usepackage{booktabs} % for much better looking tables
\usepackage{array} % for better arrays (eg matrices) in maths
\usepackage{paralist} % very flexible & customisable lists (eg. enumerate/itemize, etc.)
\usepackage{verbatim} % adds environment for commenting out blocks of text & for better verbatim
\usepackage{subfig} % make it possible to include more than one captioned figure/table in a single float
% These packages are all incorporated in the memoir class to one degree or another...
\usepackage[colorlinks=true,linkcolor=blue]{hyperref}

%%% HEADERS & FOOTERS
\usepackage{fancyhdr} % This should be set AFTER setting up the page geometry
\pagestyle{fancy} % options: empty , plain , fancy
\renewcommand{\headrulewidth}{0pt} % customise the layout...
\lhead{}\chead{}\rhead{}
\lfoot{}\cfoot{\thepage}\rfoot{}

\usepackage{longtable}

\usepackage{color}
\definecolor{pblue}{rgb}{0.13,0.13,1}
\definecolor{pgreen}{rgb}{0,0.5,0}
\definecolor{pred}{rgb}{0.9,0,0}
\definecolor{pgrey}{rgb}{0.46,0.45,0.48}



\usepackage{listings}
\lstset{% Global setting
    %language=XX,
    xleftmargin=2em,
    backgroundcolor=\color{white},
    basicstyle=\ttfamily,
    keywordstyle=\bfseries,
    commentstyle=\rmfamily\itshape,
    stringstyle=\color{pred}\ttfamily,
    flexiblecolumns,
    showspaces=false,
    showtabs=false,
    breaklines=true,
    showstringspaces=false,
    breakatwhitespace=true,
    tabsize=2,
    commentstyle=\color{pgreen}
    %numbers=left,
    %numberstyle=\footnotesize  
}


%%% SECTION TITLE APPEARANCE
\usepackage{sectsty}
\allsectionsfont{\sffamily\mdseries\upshape} % (See the fntguide.pdf for font help)
% (This matches ConTeXt defaults)

%%% ToC (table of contents) APPEARANCE
\usepackage[nottoc,notlof,notlot]{tocbibind} % Put the bibliography in the ToC
\usepackage[titles,subfigure]{tocloft} % Alter the style of the Table of Contents
\renewcommand{\cftsecfont}{\rmfamily\mdseries\upshape}
\renewcommand{\cftsecpagefont}{\rmfamily\mdseries\upshape} % No bold!

\newcommand\mgape[1]{\gape{$\vcenter{\hbox{#1}}$}}


\setmainfont{Minion Pro}
%%% END Article customizations

%%% The "real" document content comes below...
\makeindex

\title{短信管理系统}
\author{张国玉~~zgy@bz-inc.com}
%\date{} % Activate to display a given date or no date (if empty),
         % otherwise the current date is printed 

\begin{document}
\maketitle
%\makeindex
\tableofcontents
\listoffigures
\listoftables
\printindex




\part{互联网短信网关接口协议}

\begin{center}
\textbf{\huge 中国移动通信互联网短信网关接口协议}\\ \bigskip \bigskip \bigskip {\Large (China Mobile Peer to Peer, CMPP)} \\ {\large(V2.0)} \bigskip \\ \bigskip 2002年4月\\ \vfill 中国移动通信集团公司
\end{center}

\chapter{前言}

本规范为中国移动通信集团公司企业规范,简称CMPP,现阶段版本是对1.2.1版修订后形成的,为2.0版。

本规范描述了中国移动短信业务中各网元(包括ISMG、GNS和SP)之间的相关消息的类型和定义。根据业务的发展,规范中的信令操作和参数将会做进一步的调整和增加。

本规范解释权属于中国移动通信集团公司。

本规范起草单位:中国移动通信集团公司研发中心。


\chapter{范围}

本规范规定了以下三方面的内容:

\begin{compactenum}
\item 信息资源站实体与互联网短信网关之间的接口协议;
\item 互联网短信网关之间的接口协议;
\item 互联网短信网关与汇接网关之间的接口协议。
\end{compactenum}


\chapter{缩略语}

\begin{longtable}{|m{50pt}|m{140pt}|m{200pt}|}

\multicolumn{3}{r}{}
\tabularnewline\hline
英文缩写 & 英文全称 & 说明
\endhead

\caption{缩略语}\\
\hline
英文缩写 & 英文全称 & 说明
\endfirsthead

\multicolumn{3}{r}{}
\endfoot

\endlastfoot

\hline
ISMG& Internet Short Message Gateway & 互联网短信网关\\
\hline
SMPP&Short Message Peer to Peer & 短消息点对点协议\\
\hline
CMPP&China Mobile Peer to Peer& 中国移动点对点协议\\
\hline
SMC&Short Message Center&短消息中心\\
\hline
GNS&Gateway Name Server& 网关名称服务器(汇接网关)\\
\hline
SP&Service Provider& 业务提供者,即信息资源站实体\\
\hline
SMC&Short Message Control&SP为收取包月业务费用而向网关发送的消息,网关收到后不送给用户仅产生相应的话单;\\
\hline
ISMG\_Id& &网关代码:0XYZ01~0XYZ99,其中XYZ为省会区号,位数不足时左补零,如北京编号为1的网关代码为001001,江西编号为1的网关代码为079101,依此类推。\\
\hline
SP\_Id& & SP的企业代码:网络中SP地址和身份的标识、地址翻译、计费、结算等均以企业代码为依据。企业代码以数字表示,共6位,从“9XY000”至“9XY999”,其中“XY”为各移动公司代码。\\
\hline
SP\_Code & & SP的服务代码:服务代码是在使用短信方式的点播类业务中,提供给用户点播的内容/应用服务提供商代码。服务代码以数字表示,全国业务服务代码长度统一为 4 位,即“1000”-“9999”;本地业务服务代码长度统一为5 位,即  “01000”-“09999”。\\
\hline
Service\_Id &  & SP的业务类型,数字、字母和符号的 组合,由SP自定,如图片传情可定为TPCQ,股票查询可定义为11。\\
\hline
\end{longtable}


\chapter{网络结构}
\chapter{CMPP功能概述}
\chapter{协议栈}
\chapter{通信方式}
\section{长连接}
\section{短连接}
\section{本协议中涉及的端口号}
\section{交互过程中的应答方式}
\chapter{消息定义}
\section{基本数据类型}
\section{消息结构}
\section{消息头格式(Message Header)}
\section{信息资源站实体(SP)与互联网短信网关(ISMG)间的消息定义}
\subsection{SP请求连接到ISMG(CMPP\_CONNECT)操作	}
\subsubsection{CMPP\_CONNECT消息定义(SP\textrightarrow ISMG)}
\subsubsection{CMPP\_CONNECT\_RESP消息定义(ISMG\textrightarrow SP)}
\subsection{SP或ISMG请求拆除连接(CMPP\_TERMINATE)操作}
\subsubsection{CMPP\_TERMINATE消息定义(SPISMG或ISMG\textrightarrow SP)}
\subsubsection{CMPP\_TERMINATE\_RESP消息定义(SPISMG或ISMG\textrightarrow SP)}
\subsection{SP向ISMG提交短信(CMPP\_SUBMIT)操作}
\subsubsection{CMPP\_SUBMIT消息定义(SPISMG)}
\subsubsection{CMPP\_SUBMIT\_RESP消息定义(ISMG\textrightarrow SP)}
\subsection{SP向ISMG查询发送短信状态(CMPP\_QUERY)操作}
\subsubsection{CMPP\_QUERY消息的定义(SP\textrightarrow ISMG)}
\subsubsection{CMPP\_QUERY\_RESP消息的定义(ISMG\textrightarrow SP)}
\subsection{ISMG向SP送交短信(CMPP\_DELIVER)操作}
\subsubsection{CMPP\_DELIVER消息定义(ISMG\textrightarrow SP)}
\subsubsection{CMPP\_DELIVER\_RESP消息定义(SP\textrightarrow ISMG)}
\subsection{SP向ISMG发起删除短信(CMPP\_CANCEL)操作}
\subsubsection{CMPP\_CANCEL消息定义(SP\textrightarrow ISMG)}
\subsubsection{CMPP\_CANCEL\_RESP消息定义(ISMG\textrightarrow SP)}
\subsection{链路检测(CMPP\_ACTIVE\_TEST)操作}
\subsubsection{CMPP\_ACTIVE\_TEST定义(SP\textrightarrow ISMG或ISMG\textrightarrow SP)}
\subsubsection{CMPP\_ACTIVE\_TEST\_RESP定义(SP\textrightarrow ISMG或ISMG\textrightarrow SP)}
\section{互联网短信网关(ISMG)之间的消息定义}
\subsection{源ISMG请求连接到目的ISMG(CMPP\_CONNECT)操作}
\subsection{源ISMG请求拆除到目的ISMG的连接(CMPP\_TERMINATE)操作}

\subsection{链路检测(CMPP\_ACTIVE\_TEST)操作}
\subsection{源ISMG向目的ISMG转发短信(CMPP\_FWD)操作}
\subsubsection{CMPP\_FWD定义(ISMG ISMG)}
\subsubsection{CMPP\_FWD\_RESP定义(ISMG ISMG)}
\section{互联网短信网关(ISMG)与汇接网关(GNS)之间的消息定义}
\subsection{ISMG请求连接到GNS或GNS请求连接到ISMG(CMPP\_CONNECT)操作}
\subsection{ISMG请求拆除到GNS的连接或GNS请求拆除到ISMG的连接(CMPP\_TERMINATE)操作}
\subsection{ISMG向汇接网关查询MT路由(CMPP\_MT\_ROUTE)操作}
\subsubsection{CMPP\_MT\_ROUTE消息定义(ISMGGNS)}
\subsubsection{CMPP\_MT\_ROUTE\_RESP消息定义(GNS\textrightarrow ISMG)}
\subsection{ISMG向汇接网关查询MO路由(CMPP\_MO\_ROUTE)操作}
\subsubsection{CMPP\_MO\_ROUTE消息定义(ISMGGNS)}
\subsubsection{CMPP\_MO\_ROUTE\_RESP消息定义(GNS\textrightarrow ISMG)}
\subsection{ISMG向汇接网关获取路由(CMPP\_GET\_ROUTE)操作}
\subsubsection{CMPP\_GET\_ ROUTE消息定义(ISMG\textrightarrow GNS)}
\subsubsection{CMPP\_GET\_ ROUTE\_RESP消息定义(GNS\textrightarrow ISMG)}
\subsection{ISMG向汇接网关更新MT路由(CMPP\_MT\_ROUTE\_UPDATE)操作}
\subsubsection{CMPP\_MT\_ROUTE\_UPDATE消息定义(ISMGGNS)}
\subsubsection{CMPP\_MT\_ROUTE\_UPDATE\_RESP消息定义(GNS\textrightarrow ISMG)}
\subsection{ISMG向汇接网关更新MO路由(CMPP\_MO\_ROUTE\_UPDATE)操作}
\subsubsection{CMPP\_MO\_ROUTE\_UPDATE消息定义(ISMGGNS)}
\subsubsection{CMPP\_MO\_ROUTE\_UPDATE\_RESP消息定义(GNS\textrightarrow ISMG)}
\subsection{汇接网关向ISMG更新MT路由(CMPP\_PUSH\_MT\_ROUTE\_UPDATE)操作}
\subsubsection{CMPP\_PUSH\_MT\_ROUTE\_UPDATE消息定义(GNS\textrightarrow ISMG)}
\subsubsection{CMPP\_PUSH\_MT\_ROUTE\_UPDATE\_RESP消息定义(ISMG\textrightarrow GNS)}
\subsection{汇接网关向ISMG更新MO路由(CMPP\_PUSH\_MO\_ROUTE\_UPDATE)操作}
\subsubsection{CMPP\_PUSH\_MO\_ROUTE\_UPDATE消息定义(GNS\textrightarrow ISMG)}
\subsubsection{CMPP\_PUSH\_MO\_ROUTE\_UPDATE\_RESP消息定义(ISMG\textrightarrow GNS)}
\section{系统定义}
\subsection{Command\_Id定义}
\section{MO状态报告的产生}
\section{修订历史}



\end{document}
